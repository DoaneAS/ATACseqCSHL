\documentclass[]{article}
\usepackage{lmodern}
\usepackage{amssymb,amsmath}
\usepackage{ifxetex,ifluatex}
\usepackage{fixltx2e} % provides \textsubscript
\ifnum 0\ifxetex 1\fi\ifluatex 1\fi=0 % if pdftex
  \usepackage[T1]{fontenc}
  \usepackage[utf8]{inputenc}
\else % if luatex or xelatex
  \ifxetex
    \usepackage{mathspec}
  \else
    \usepackage{fontspec}
  \fi
  \defaultfontfeatures{Ligatures=TeX,Scale=MatchLowercase}
  \newcommand{\euro}{€}
\fi
% use upquote if available, for straight quotes in verbatim environments
\IfFileExists{upquote.sty}{\usepackage{upquote}}{}
% use microtype if available
\IfFileExists{microtype.sty}{%
\usepackage{microtype}
\UseMicrotypeSet[protrusion]{basicmath} % disable protrusion for tt fonts
}{}


\setlength{\parindent}{0pt}
\setlength{\parskip}{6pt plus 2pt minus 1pt}
\setlength{\emergencystretch}{3em}  % prevent overfull lines
\providecommand{\tightlist}{%
  \setlength{\itemsep}{0pt}\setlength{\parskip}{0pt}}
\setcounter{secnumdepth}{5}

%%% Use protect on footnotes to avoid problems with footnotes in titles
\let\rmarkdownfootnote\footnote%
\def\footnote{\protect\rmarkdownfootnote}

%%% Change title format to be more compact
\usepackage{titling}

\RequirePackage[]{/Library/Frameworks/R.framework/Versions/3.3/Resources/library/BiocStyle/resources/tex/Bioconductor2}

% Create subtitle command for use in maketitle
\newcommand{\subtitle}[1]{
  \posttitle{
    \begin{center}\large#1\end{center}
    }
}

\setlength{\droptitle}{-2em}
  \title{Bioconductor style for PDF documents}
  \pretitle{\vspace{\droptitle}\centering\huge}
  \posttitle{\par}
  \author{Andrzej Oleś\thanks{\ttfamily\href{mailto:andrzej.oles@embl.de}{\nolinkurl{andrzej.oles@embl.de}}}}
  \author{Wolfgang Huber}
  \affil{European Molecular Biology Laboratory, Heidelberg}  \author{Martin Morgan}
  \affil{Roswell Park Cancer Institute, Bufallo}  \preauthor{\centering\large\emph}
  \postauthor{\par}
  \predate{\centering\large\emph}
  \postdate{\par}
  \date{17 November 2016}


% Redefines (sub)paragraphs to behave more like sections
\ifx\paragraph\undefined\else
\let\oldparagraph\paragraph
\renewcommand{\paragraph}[1]{\oldparagraph{#1}\mbox{}}
\fi
\ifx\subparagraph\undefined\else
\let\oldsubparagraph\subparagraph
\renewcommand{\subparagraph}[1]{\oldsubparagraph{#1}\mbox{}}
\fi

% code highlighting
\definecolor{fgcolor}{rgb}{0.251, 0.251, 0.251}
\newcommand{\hlnum}[1]{\textcolor[rgb]{0.816,0.125,0.439}{#1}}%
\newcommand{\hlstr}[1]{\textcolor[rgb]{0.251,0.627,0.251}{#1}}%
\newcommand{\hlcom}[1]{\textcolor[rgb]{0.502,0.502,0.502}{\textit{#1}}}%
\newcommand{\hlopt}[1]{\textcolor[rgb]{0,0,0}{#1}}%
\newcommand{\hlstd}[1]{\textcolor[rgb]{0.251,0.251,0.251}{#1}}%
\newcommand{\hlkwa}[1]{\textcolor[rgb]{0.125,0.125,0.941}{#1}}%
\newcommand{\hlkwb}[1]{\textcolor[rgb]{0,0,0}{#1}}%
\newcommand{\hlkwc}[1]{\textcolor[rgb]{0.251,0.251,0.251}{#1}}%
\newcommand{\hlkwd}[1]{\textcolor[rgb]{0.878,0.439,0.125}{#1}}%
\let\hlipl\hlkwb
%
\usepackage{fancyvrb}
\newcommand{\VerbBar}{|}
\newcommand{\VERB}{\Verb[commandchars=\\\{\}]}
\DefineVerbatimEnvironment{Highlighting}{Verbatim}{commandchars=\\\{\}}
%
\newenvironment{Shaded}{\begin{myshaded}}{\end{myshaded}}
% set background for result chunks
\let\oldverbatim\verbatim
\renewenvironment{verbatim}{\color{codecolor}\begin{myshaded}\begin{oldverbatim}}{\end{oldverbatim}\end{myshaded}}
%
\newcommand{\KeywordTok}[1]{\hlkwd{#1}}
\newcommand{\DataTypeTok}[1]{\hlkwc{#1}}
\newcommand{\DecValTok}[1]{\hlnum{#1}}
\newcommand{\BaseNTok}[1]{\hlnum{#1}}
\newcommand{\FloatTok}[1]{\hlnum{#1}}
\newcommand{\CharTok}[1]{\hlstr{#1}}
\newcommand{\StringTok}[1]{\hlstr{#1}}
\newcommand{\CommentTok}[1]{\hlcom{#1}}
\newcommand{\OtherTok}[1]{{#1}}
\newcommand{\AlertTok}[1]{\textcolor[rgb]{0.94,0.16,0.16}{{#1}}}
\newcommand{\FunctionTok}[1]{\textcolor[rgb]{0.00,0.00,0.00}{{#1}}}
\newcommand{\RegionMarkerTok}[1]{{#1}}
\newcommand{\ErrorTok}[1]{\textbf{{#1}}}
\newcommand{\NormalTok}[1]{\hlstd{#1}}
%
\AtBeginDocument{\bibliographystyle{/Library/Frameworks/R.framework/Versions/3.3/Resources/library/BiocStyle/resources/tex/unsrturl}}

\begin{document}
\maketitle
\begin{abstract}
Instructions on enabling \emph{Bioconductor} style in your \emph{R}
markdown vignettes.
\end{abstract}

\packageVersion{BiocStyle 2.2.0}

{
\setcounter{tocdepth}{2}
\tableofcontents
\newpage
}
\section{Authoring R markdown PDF package
vignettes}\label{authoring-r-markdown-pdf-package-vignettes}

To enable the \emph{Bioconductor} style in your markdown (\texttt{.Rmd})
vignettes you need to:

\begin{itemize}
\item
  Edit to the \texttt{DESCRIPTION} file by adding

\begin{verbatim}
VignetteBuilder: knitr
Suggests: BiocStyle, knitr, rmarkdown
\end{verbatim}
\item
  Specify \texttt{BiocStyle::pdf\_document} as the output format and add
  vignette metadata in the document header:

\begin{verbatim}
---
title: "Vignette Title"
author: "Vignette Author"
output: 
  BiocStyle::pdf_document
vignette: >
  %\VignetteIndexEntry{Vignette Title}
  %\VignetteEngine{knitr::rmarkdown}
  %\VignetteEncoding{UTF-8}  
---
\end{verbatim}
\end{itemize}

The \texttt{vignette} section is required in order to instruct R how to
build the vignette. Note that the
\texttt{\textbackslash{}VignetteIndexEntry} should match the
\texttt{title} of your vignette.

It is also possible to specify additional details such as:

\begin{itemize}
\tightlist
\item
  document compilation \texttt{date} inserted using the function
  \texttt{doc\_date}
\item
  \texttt{package} version specification added by \texttt{pkg\_ver}
\item
  vignette \texttt{abstract}
\end{itemize}

as in the following example.

\begin{verbatim}
---
title: "Vignette Title"
author: "Vignette Author"
date: "`r doc_date()`"
package: "`r pkg_ver('BiocStyle')`"
abstract: >
  Vignette Abstract
vignette: >
  %\VignetteIndexEntry{Vignette Title}
  %\VignetteEngine{knitr::rmarkdown}
  %\VignetteEncoding{UTF-8}
output: 
  BiocStyle::pdf_document
---
\end{verbatim}

\section{Style macros}\label{style-macros}

\emph{\href{http://bioconductor.org/packages/BiocStyle}{BiocStyle}}
introduces the following macros useful when referring to \emph{R}
packages:

\begin{itemize}
\item
  \texttt{`r\ Biocpkg("IRanges")`} for \emph{Bioconductor} software,
  annotation and experiment data packages, including a link to the
  release landing page or if the package is only in devel, to the devel
  landing page,
  \emph{\href{http://bioconductor.org/packages/IRanges}{IRanges}}.
\item
  \texttt{`r\ CRANpkg("data.table")`} for \emph{R} packages available on
  CRAN, including a link to the FHCRC CRAN mirror landing page,
  \emph{\href{http://cran.fhcrc.org/web/packages/data.table/index.html}{data.table}}.
\item
  \texttt{`r\ Githubpkg("rstudio/rmarkdown")`} for \emph{R} packages
  available on GitHub, including a link to the package repository,
  \emph{\href{https://github.com/rstudio/rmarkdown}{rmarkdown}}.
\item
  \texttt{`r\ Rpackage("MyPkg")`} for \emph{R} packages that are
  \emph{not} available on \emph{Bioconductor}, CRAN or GitHub;
  \emph{MyPkg}.
\end{itemize}

\section{Session info}\label{session-info}

Here is the output of \texttt{sessionInfo()} on the system on which this
document was compiled:

\begin{verbatim}
## R version 3.3.1 (2016-06-21)
## Platform: x86_64-apple-darwin13.4.0 (64-bit)
## Running under: OS X 10.12.1 (Sierra)
## 
## locale:
## [1] en_GB.UTF-8/en_GB.UTF-8/en_GB.UTF-8/C/en_GB.UTF-8/en_GB.UTF-8
## 
## attached base packages:
## [1] stats     graphics  grDevices utils     datasets  methods  
## [7] base     
## 
## other attached packages:
## [1] BiocStyle_2.2.0      BiocInstaller_1.24.0
## 
## loaded via a namespace (and not attached):
##  [1] magrittr_1.5       assertthat_0.1     tools_3.3.1       
##  [4] htmltools_0.3.5    yaml_2.1.14        tibble_1.2        
##  [7] Rcpp_0.12.7        stringi_1.1.2      rmarkdown_1.1     
## [10] knitr_1.15         stringr_1.1.0.9000 digest_0.6.10     
## [13] evaluate_0.10
\end{verbatim}

\end{document}
